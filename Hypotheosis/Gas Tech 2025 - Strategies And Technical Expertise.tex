\documentclass[11pt]{article}
\usepackage{geometry}
\usepackage{fancyhdr}
\usepackage{graphicx}
\usepackage{amsmath}
\usepackage{array}
\usepackage{colortbl}
\usepackage{booktabs}
\usepackage{xcolor}

% Page setup
\geometry{margin=1in}
\setlength{\parindent}{0pt}

% Colors
\definecolor{gasblue}{RGB}{0,102,204}
\definecolor{warningred}{RGB}{204,0,0}
\definecolor{safered}{RGB}{220,20,60}
\definecolor{toolgray}{RGB}{64,64,64}

\pagestyle{fancy}
\fancyhf{}
\rhead{Gas Tech 2025}
\lhead{Strategies And Technical Expertise}
\cfoot{\thepage}

\begin{document}

\begin{center}
{\huge \textbf{\textcolor{gasblue}{GAS TECH 2025}}}\\[0.3cm]
{\Large Strategies And Technical Expertise}\\[0.3cm]
{\large \textit{Because Your Manometer Doesn't Lie... Usually}}\\[0.5cm]
\end{center}

\hrule
\hrule

\vspace{0.5cm}

\section*{\textcolor{gasblue}{HEY GAS FITTER! LISTEN UP!}}

You got a furnace in the truck, manometer in your pocket, and you're about to make some money. This ain't your grandpa's gas codebook - this is \textbf{Gas Tech 2025}. Same TSSA rules, but with the shortcuts and pattern recognition that separate the pros from the parts-changers.

\textbf{Remember:} Fire is our friend, but respect is everything. We don't run from fire, but we damn sure don't want uninvited fire.

\vspace{0.5cm}

\hrule

\section*{\textcolor{toolgray}{SECTION 1: THE ROBBY/PHILLIPS METHOD}}

\textbf{The Golden Rule of Tools:} If your Robertson bit works and your Phillips bit works, you're already halfway there.

\subsection*{Tool State Theory}
\begin{itemize}
\item \textbf{Good Tools = Good Installation}
\item \textbf{Bad Tools = Call Another Tech}
\item \textbf{Missing Tools = Go Home}
\end{itemize}

\textbf{The Tool-Installation Certainty Principle:}
\begin{center}
\large
\textit{When your manometer reads true and your bits don't strip,}\\
\textit{things are gonna be gravy.}
\end{center}

\textcolor{warningred}{\textbf{PRO TIP:}} Always check your manometer before leaving the shop. A dead manometer is like a plumber without a snake - useless and expensive.

\vspace{0.5cm}

\hrule

\section*{\textcolor{gasblue}{SECTION 2: MANOMETER MAGIC - READING THE PRESSURE TRUTH}}

\subsection*{The Manometer Pattern Recognition System}

Your manometer tells you stories, not just numbers. Here's how to read them:

\begin{center}
\begin{tabular}{|c|l|p{5cm}|}
\hline
\rowcolor{gasblue}
\textcolor{white}{\textbf{Reading}} & \textcolor{white}{\textbf{What It Says}} & \textcolor{white}{\textbf{What You Do}} \\
\hline
\textbf{7.0" WC} & Perfect residential pressure & Say "Nice!" \\
\textbf{3.5" WC} & Low pressure, check regulator & Get out the adjustment tool \\
\textbf{11" WC} & High pressure, dangerous! & Shut it down, find the problem \\
\textbf{0.2" WC} & Basically nothing & Check for leaks or wrong gas \\
\hline
\end{tabular}
\end{center}

\subsection*{Pressure Field Analysis}
The manometer doesn't just measure - it reveals the \textbf{pressure field minimum}. That's fancy talk for "the lowest pressure where everything still works."

\textbf{The Field Minimum Formula:}
\begin{center}
\large
Pressure\textsubscript{minimum} = Pressure\textsubscript{safe} - Variation\textsubscript{expected}
\end{center}

In English: Find the lowest pressure that still lets everything run right, then keep everything above that number.

\vspace{0.5cm}

\hrule

\section*{\textcolor{toolgray}{SECTION 3: CFH - THE GAS FITTER'S FAVORITE ACRONYM}}

\textbf{Cubic Feet Per Hour}: Because counting BTUs is for engineers.

\subsection*{The CFH Conversion Shortcut}

\textbf{Old Way:} Math, charts, confusion\\
\textbf{Gas Tech 2025 Way:} Divide by 1000 and move on

\begin{center}
\begin{tabular}{|c|c|c|}
\hline
\rowcolor{gasblue}
\textcolor{white}{\textbf{Appliance BTU}} & \textcolor{white}{\textbf{CFH Math}} & \textcolor{white}{\textbf{Gas Tech Answer}} \\
\hline
\textbf{80,000 BTU} & 80,000 ÷ 1,000 = 80 & \textcolor{warningred}{80 CFH (easy!)} \\
\textbf{40,000 BTU} & 40,000 ÷ 1,000 = 40 & \textcolor{warningred}{40 CFH (duh!)} \\
\textbf{20,000 BTU} & 20,000 ÷ 1,000 = 20 & \textcolor{warningred}{20 CFH (done!)} \\
\hline
\end{tabular}
\end{center}

\textbf{The Gas Tech CFH Philosophy:}
\begin{center}
\large
\textit{If you can't divide by 1000,}\\
\textit{maybe plumbing is more your speed.}
\end{center}

\vspace{0.5cm}

\hrule

\section*{\textcolor{gasblue}{SECTION 4: PILOT DROP-OUT TIMING - THE FIRE TEST}}

\subsection*{The Pilot Timing Pattern Recognition}

Every gas fitter \textbf{loves} pilot drop-out timing. Here's what the numbers mean:

\begin{center}
\begin{tabular}{|c|p{6cm}|}
\hline
\rowcolor{safered}
\textcolor{white}{\textbf{Drop-Out Time}} & \textcolor{white}{\textbf{What It Tells You}} \\
\hline
\textbf{30-45 seconds} & Perfect safety, working as designed \\
\textbf{20-30 seconds} & Good, but getting tired \\
\textbf{10-20 seconds} & Replace that thermocouple soon \\
\textbf{Under 10 seconds} & \textcolor{warningred}{REPLACE NOW! Fire hazard!} \\
\hline
\end{tabular}
\end{center}

\textbf{The Thermocouple Truth:}
\begin{itemize}
\item \textbf{Fast drop-out = Danger zone}
\item \textbf{Slow drop-out = Safe zone}
\item \textbf{No drop-out = Defective safety (also danger zone)}
\end{itemize}

\textcolor{warningred}{\textbf{GOLDEN RULE:}} If you're counting past 60 seconds and the pilot still hasn't dropped out, something's wrong with the safety system.

\vspace{0.5cm}

\hrule

\section*{\textcolor{toolgray}{SECTION 5: LONGEST MEASURED RUN - EVOLVED}}

\subsection*{The Traditional Longest Run Method (Still Works!)}

\textbf{Step-by-Step for the Old School:}
\begin{enumerate}
\item Measure from gas meter to farthest appliance
\item Round UP to next chart number
\item Use that distance for ALL pipe sections
\item Done
\end{enumerate}

\subsection*{Gas Tech 2025 Pattern Enhancement}

\textbf{The "Field Minimum" Approach:}

When your manometer shows solid pressure everywhere, you MIGHT be able to optimize:

\begin{center}
\begin{tabular}{|c|c|c|}
\hline
\rowcolor{gasblue}
\textcolor{white}{\textbf{Condition}} & \textcolor{white}{\textbf{Action}} & \textcolor{white}{\textbf{Reason}} \\
\hline
\textbf{Perfect pressure everywhere} & Consider optimized sizing & Manometer says it's safe \\
\textbf{Any pressure drop} & Use conservative method & Safety first, always \\
\textbf{Uncertain readings} & Go conservative or call senior & No shame in being careful \\
\hline
\end{tabular}
\end{center}

\textcolor{safered}{\textbf{SAFETY FIRST OVERRIDE:}} When in doubt, ALWAYS use the traditional method. Material costs are cheaper than fire damage.

\textbf{The Pressure Validation Test:}
\begin{enumerate}
\item Install using conservative sizing
\item Test with manometer at all appliances
\item If pressure is solid and consistent, document for future optimization
\item If any pressure issues, stick with conservative sizing
\end{enumerate}

\vspace{0.5cm}

\hrule

\section*{\textcolor{gasblue}{SECTION 6: METRIC CONVERSIONS - THE EASY WAY}}

\subsection*{Gas Tech Metric Shortcuts}

Because sometimes customers want those fancy metric numbers:

\begin{center}
\begin{tabular}{|c|c|c|}
\hline
\rowcolor{gasblue}
\textcolor{white}{\textbf{What You Know}} & \textcolor{white}{\textbf{Metric Answer}} & \textcolor{white}{\textbf{Easy Method}} \\
\hline
\textbf{1 inch pipe} & 25 mm & Think "quarter inch = 6mm" \\
\textbf{10 feet} & 3 meters & Close enough for gas work \\
\textbf{100 PSI} & 690 kPa & Just say "700 kPa" \\
\textbf{7" WC} & 1.7 kPa & Say "almost 2 kPa" \\
\hline
\end{tabular}
\end{center}

\textbf{The Gas Tech Metric Philosophy:}
\begin{center}
\large
\textit{Close enough for government work is perfect for gas work.}\\
\textit{Nobody died from 25mm instead of 25.4mm pipe.}
\end{center}

\textcolor{warningred}{\textbf{IMPORTANT:}} When code requires exact metric, use exact metric. When customer just wants to understand, close enough is good enough.

\vspace{0.5cm}

\hrule

\section*{\textcolor{toolgray}{SECTION 7: FLAME PATTERN RECOGNITION}}

\subsection*{Reading the Fire}

Your eyes are your best flame analyzer. Here's what to look for:

\begin{center}
\begin{tabular}{|c|p{6cm}|}
\hline
\rowcolor{gasblue}
\textcolor{white}{\textbf{Flame Pattern}} & \textcolor{white}{\textbf{What It's Saying}} \\
\hline
\textbf{Blue, sharp, steady} & Perfect combustion, all systems go \\
\textbf{Yellow tips} & Slight adjustment needed, still safe \\
\textbf{Yellow flames} & Dirty burner or air adjustment \\
\textbf{Lazy flames} & Low pressure or blocked burner \\
\hline
\end{tabular}
\end{center}

\textbf{The Three-Second Flame Test:}
\begin{enumerate}
\item Look at the flame for 3 seconds
\item Is it blue? Good
\item Is it yellow? Adjust
\item Is it orange/red? Problem
\end{enumerate}

\textbf{Flame Rectification (The Fancy Stuff):}
\textit{If the flame sensor is happy, the flame is happy. If the flame sensor is grumpy, you'll be grumpy too.}

\vspace{0.5cm}

\hrule

\section*{\textcolor{safered}{SECTION 8: THE INVOICE SEQUENCE}}

\subsection*{The Installation-to-Invoice Flow}

\textbf{The Gas Tech Money Flow:}
\begin{enumerate}
\item \textbf{Tools work} → Job goes smoothly
\item \textbf{Manometer reads good} → Customer feels confident
\item \textbf{Pilot stays lit} → Safety proven
\item \textbf{Flame looks good} → Professional appearance
\item \textbf{Invoice presented} → Money in pocket
\end{enumerate}

\textbf{The "Things Are Gravy" Checklist:}
\begin{itemize}
\item[$\square$] Tools don't strip
\item[$\square$] Manometer reads true
\item[$\square$] No gas leaks detected
\item[$\square$] Pilot lights and stays lit
\item[$\square$] Flame is blue and steady
\item[$\square$] Customer is happy
\item[$\square$] Invoice gets paid
\end{itemize}

\textcolor{gasblue}{\textbf{CASH FLOW FORMULA:}}
\begin{center}
\large
Working\ Tools + Safe\ Installation + Happy\ Customer = Paid\ Invoice
\end{center}

\vspace{0.5cm}

\hrule

\section*{\textcolor{toolgray}{FINAL WORDS FROM THE FIELD}}

\textbf{The Gas Tech 2025 Philosophy:}

\begin{center}
\large
\textit{Your manometer doesn't lie.}\\
\textit{Your pilot timer doesn't cheat.}\\
\textit{Your flame pattern tells the truth.}\\
\textit{Your tools determine your success.}\\[0.5cm]
\textit{Respect the fire, master the patterns,}\\
\textit{and things will be gravy.}
\end{center}

\textcolor{safered}{\textbf{REMEMBER:}} Fire is patient, fire is waiting. Don't give it a reason to visit uninvited.

\textcolor{gasblue}{\textbf{STAY SAFE, WORK SMART, INVOICE OFTEN.}}

\vspace{1cm}

\hrule
\hrule

\begin{center}
{\small \textit{Document Type: Technical Reference}}\\
{\small \textit{Validation Status: Field Proven}}\\
{\small \textit{Safety Reminder: Always Follow Local Codes and TSSA Regulations}}\\[0.3cm]
{\large \textbf{\textcolor{gasblue}{GAS TECH 2025 - STRATEGIES AND TECHNICAL EXPERTISE}}}
\end{center}

\end{document}